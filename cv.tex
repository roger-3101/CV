\documentclass[a4paper]{../template} % a4paper for A4

%----------------------------------------------------------------------------------------
%	 PERSONAL INFORMATION
%----------------------------------------------------------------------------------------

% If you don't need one or more of the below, just remove the content leaving the command, e.g. \cvnumberphone{}

\profilepic{} % Profile picture

\cvname{Romain\\ Pastorelli} % Your name
\cvjobtitle{Étudiant} % Job title/career

\cvdate{} % Date of birth
\cvaddress{} % Short address/location, use \newline if more than 1 line is required
\cvnumberphone{07-82-62-76-61} % Phone number
\cvsite{}%https://romain-pastorelli.github.io/CV/index.html} % Personal website
\cvmail{pastorelliromain@gmail.com} % Email address

%----------------------------------------------------------------------------------------

\begin{document}

%----------------------------------------------------------------------------------------
%	 ABOUT ME
%----------------------------------------------------------------------------------------

\aboutme{Passionné d'informatique et de sciences en général, je suis très curieux et j'aime m'informer, acquérir de nombreuses connaissances et développer de nouvelles compétences. Depuis toujours j'appends en autodidactie donc j'ai beaucoup développé mon autonomie.\\
Je suis capable de prendre des initiatives lorsque c'est nécessaire, je m'adapte rapidemment et j'apprends vite. Je sais travailler aussi bien seul qu'en équipe} % To have no About Me section, just remove all the text and leave \aboutme{}

%----------------------------------------------------------------------------------------
%	 SKILLS
%----------------------------------------------------------------------------------------

% Skill bar section, each skill must have a value between 0 an 6 (float)
\skills{{Adaptabilité/5},{Travail en équipe/5},{Anglais/5},{Français/6}}

%----------------------------------------------------------------------------------------
%	 COMPUTER SCIENCE
%----------------------------------------------------------------------------------------

\prog{{Java/4},{Shell Bash/5},{Emacs/5},{Latex, Org Mode, Markdown/5},{HTML5, CSS3, JS/5},{C/5},{Python/5.5}}

\makeprofile % Print the sidebar

%----------------------------------------------------------------------------------------
%	 EXPERIENCE
%----------------------------------------------------------------------------------------

\section{Parcours professionnel}

\begin{twenty} % Environment for a list with descriptions
	\twentyitem{été 2023\\08/2022}{Runner, dans la restauration}{Le Campa, Aix-les-Bains, 73100\\L'arbre à Palabres, Aix-les-Bains, 73100}{En tant que runner dans une équipe d'environ 15 personnes, mon travail était majoritairement constitué de préparation de salle avant les services, de service de plats et de débarassage et rangement de la salle en fin de services. J'ai aussi eu l'occasion de faire un peu de plonge et d'aider le chef cuisinier. Grâce à cette expérience j'ai pu améliorer ma réactivité et mon endurance.}
	\twentyitem{07/2021-22}{Agent de production agroalimentaire}{Rochias, Issoire, 63500}{J'ai travaillé dans une équipe de 6 personnes, sur la chaîne de production. Mon rôle consistait à surveiller les machines de transformation, manutentionner des sacs ou caisses d'ingrédients, réaliser des recettes et préparer les commandes à l'export.}
	\twentyitem{2018 - 2021}{Professeur particulier en mathématique}{Domicile de particuliers, Aix-les-Bains, 73100}{Je donnais hebdomadairement des cours de mathématique à des élèves, allant du niveau CM1 à la Première. C'était en majorité de l'aide au devoirs ou des révisions de cours en prévision de contrôles mais j'ai aussi eu l'occasion de créer, pour un enfant, un vrai programme d'apprentissage, s'étallant sur 2 mois, à base de leçons, d'exercices, de petites interro fréquentes et d'un contôle final. Cela m'a permis de développer ma pédagogie, ma patience et mon organisation de travail.}
	%\twentyitem{<dates>}{<title>}{<location>}{<description>}
\end{twenty}

%----------------------------------------------------------------------------------------
%	 EDUCATION
%----------------------------------------------------------------------------------------

\section{Formation}

\begin{twenty} % Environment for a list with descriptions
	\twentyitem{2023 - actuel}{Ecole d'ingénieur en informatique}{ISIMA - Clermont-Ferrand}{Je travaille en particulier sur de la modélisation mathématique et/ou physique, mais aussi sur la résolution de problème de manière informatisée (programmation C et Scheme). Je m'intéresse particulèrement à la cyber-sécurité. Je participe également au projet Handi-Tutorat pour lequel j'accompagne un lycéen en situation de handicap et l'aide à passer le "mur du BAC". J'ai également fait du tutorat pour des étudiant·e·s de L1.}
	\twentyitem{2021 - 2023}{Classe Préparatoire Intégrée Prép'ISIMA}{ISIMA - Clermont-Ferrand}{J'y ai développé mes compétences en informatique et en mathématiques, en équipe ou individuellement. J'ai eu le temps de me lancer dans quelques projets en dehors des cours comme des challenges sur la plateforme Root-Me. J'ai aussi suivi une grande partie des cours à la Prépa des INP, ainsi que quelques cours en Psychologie, en parallèle de ma formation, sur mon temps libre.}
	\twentyitem{2018 - 2021}{Baccalauréat Général}{Lycée Général et Technologique Louis Armand - Chambéry}{J'ai obtenu le BAC mention Bien, avec les spécialités Mathémathique et Sciences de l'Ingénieur et l'option Math Expertes. J'ai aussi participé au club Math en Jeans de mon lycée. La première année portait que la cryptographie et la seconde, sur le déplacement des continents plastiques autour du globe.}
	%\twentyitem{<dates>}{<title>}{<location>}{<description>}
\end{twenty}

%----------------------------------------------------------------------------------------
%	 OTHER INFORMATION
%----------------------------------------------------------------------------------------

%\section{Other information}

%----------------------------------------------------------------------------------------

\end{document} 
